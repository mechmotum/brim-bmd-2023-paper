\documentclass{bmd2023p}

\newcommand{\code}[1]{\texttt{#1}}
\newcommand{\softwarepackage}[1]{\textit{#1}}
\newcommand{\sympy}{\softwarepackage{SymPy}}
\newcommand{\package}{\softwarepackage{BRiM}}
\newcommand{\pycollo}{\softwarepackage{Pycollo}}
\newcommand{\whypackagename}{\textit{B}icycle-\textit{Ri}der \textit{M}odels}

\newcommand{\thesisreference}[1]{\textit{#1}}

\usepackage{todonotes}
\usepackage{url}

\begin{document}

%% Titlepage variables

% Title
\title{BRiM: A Modular Bicycle-Rider Modeling Framework}

% List authors for the title. Call \addauthor{name}{affiliationID} for every
% author. Authors will appear in the order off calls to \addauthor. Please
% manually specify the correct affiliation ID and an asterisk to the
% corresponding author
\addauthor{Timótheüs J. Stienstra}{1,*}
\addauthor{Samuel G. Brockie}{1}
\addauthor{Jason K. Moore}{1}

% List authors with only initials for given names, displays only in footer.
\authorfooter{Stienstra, T. J., Brockie, S. G., \& Moore, J. K.}

% List affiliations. Call \addaffiliation{id}{name}{EmailOrcidString} once for
% every distinct institution, where id specifies the affiliation ID (ensure
% that this corresponds to the IDs used with \addauthor), name specifies the
% affiliation name, and EmailOrcidString is a string listing emails and ORCIDs
% (optional) affiliated with this affiliation separating mail and ORCID using a
% comma and different author credentials using a semicolon.

\addaffiliation{1}{Faculty of Mechanical, Maritime and Materials Engineering (3mE), Delft University of Technology, The Netherlands}{t.j.stienstra@student.tudelft.nl, ORCID 0000-0003-4600-515X}

% The following variables will be updated by the publisher, do not edit.
\doi{XX.XXXX/XX.XXXX}
\year{2023}
\editor{Firstname Lastname}
\submitteddate{dd/mm/yyyy}
\accepteddate{dd/mm/yyyy}
\publisheddate{dd/mm/yyyy}
\citation{}
\issn{2667-2812}
% End publisher variables.

%% End titlepage variables

\maketitle

\section*{Abstract:}

\todo[inline]{Write abstract once rest of paper is drafted}

\newpage
\pagestyle{headings}

\section*{Introduction}

\todo[inline]{Paragraph: Explanation of the gap}
\thesisreference{introduction paragraph 2 and 3}

\todo[inline]{Paragraph: Summarize requirements from Timo's thesis}
\thesisreference{I phrased a summarized version implicitly in the introduction where I explain the goal}

\todo[inline]{Emphasize the aim/objectives}
\thesisreference{Introduction paragraph 3-4}

\todo[inline]{Paragraph: Overview of the paper/how aim and objectives are going to be met}

\section{BRiM Method}

\todo[inline]{Paragraph: three levels of description when using BRiM, where the low equation level and intermediate bodies and joints level are within SymPy and BRiM wraps around it adding a high component level \thesisreference{introduction paragraph 3; opening CH3 paragraph 1-2}. Immediately also mention Kane's method \thesisreference{why see 2.2 paragraph 2 and appendix A paragraph 1}}

\subsection{BRiM Core}

\begin{itemize}
    \item Usage of Kane's method
    \item Understand the aggregation of models
    \item The translation towards the core components of BRiM: \textit{The models are the main components, where each model describes a system or subsystem following the tree structure explained in section 3.3.1. The connections can be seen as a utility of parent models to describe a modular interaction between submodels resembling many characteristics with the connectors from the flat graph structure. The load groups are predefined sets of actuators and loads, which are commonly associated with a specific model or connection.} \thesisreference{CH 3.4; CH 3.6 summary; abstract paragraph 3}
    \item chapter 4
\end{itemize}

\subsection{BRiM Models}

\begin{itemize}
    \item
\end{itemize}


\section{Results Method}

\subsection{Demonstrations/Experiments}

\begin{itemize}
    \item Just the Whipple bicycle (base case) with steer actuator, three different frame geometries (which ones? \url{http://moorepants.github.io/dissertation/physicalparameters.html#bicycle-parameters})
    \item front frame suspension for the mountain bike parametrization
    \item Upper-body bicycle-rider with steer actuator
    \item Upper-body bicycle-rider with elbow torques (no steer torque)
\end{itemize}

\subsection{Simulation/Optimization}

\begin{itemize}
    \item Generate model
    \item Conduct forward simulation with P-controller to generate an initial guess
    \item Solve OCP using opty
\end{itemize}

\section{Results}

\section{Discussion}

\section*{Conclusion}

\todo[inline]{Summarize conclusions from Timo's thesis}

\todo[inline]{Add acknowledgements etc., specifically mention CZI with reference to funding using specific wording}
This project has been made possible in part by CZI grant CZIF2021-006198 and grant DOI https://doi.org/10.37921/240361looxoj from the Chan Zuckerberg Initiative Foundation (funder DOI 10.13039/100014989).


\bibliographystyle{apalike}
\bibliography{references.bib}

\end{document}
